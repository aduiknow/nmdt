\documentclass[10pt]{article}
\usepackage{pictex, latexsym, graphicx,amsmath,amssymb,amsbsy,amsfonts,amsthm,verbatim}
\usepackage{graphics}
\usepackage{fullpage}
\usepackage{fancyhdr}
\usepackage{enumerate}
\usepackage{algorithm,algorithmic}
\usepackage{multirow}

\setlength{\voffset}{-0.25in}
\setlength{\headsep}{0.25in}
\setlength{\parskip}{1em}
\setlength{\parindent}{0em}
\newcounter{problem}
\newcommand{\problem}{\textbf{\refstepcounter{problem}Problem \theproblem.} }

% These are macro shortcuts! Helpful!
\def\vu{\mathbf{u}}
\def\vx{\mathbf{x}}
\def\vb{\mathbf{b}}
\def\vv{\mathbf{v}}
\def\vw{\mathbf{w}}

\renewcommand{\implies}{\rightarrow}
\renewcommand{\lor}{\vee}
\renewcommand{\land}{\wedge}
\newcommand{\xor}{\oplus}
\renewcommand{\iff}{\leftrightarrow}
\newcommand{\TRUE}{\mathbf{T}}
\newcommand{\FALSE}{\mathbf{F}}
\newcommand{\universe}{\mathcal{U}}

\begin{document}
\pagestyle{fancyplain}
\chead{DISCRETE STRUCTURE (CO1007) --- SOLUTION to Homework02a --- PREDICATE LOGIC}
\textit{\textbf{\underline{Instruction:}} Type your answers to the following question by LaTeX and submit a zipped file (included.pdf and .tex files) to E-learning site individually. One page per problem. Please use the solution template provided.}
\textit{\textbf{Ref: Rosen's 7th ed - Sec 1.4-1.6}}\\
\textbf{Problem 1.} [5pts] Let P(x) be the statement "$1/x$ =$x$". If the domain consists of real numbers, then what are these truth values?\\
  \begin{enumerate}[a.]
    \item $P(2)$.
    \item $P(1)$.
    \item $\exists xP(X)$
    \item $\forall x\ne 0(P(x))$
  \end{enumerate}
\textit{\textbf{Solution:}}
  \begin{enumerate}[a.]
    \item $P(2)$ :F
    \item $P(1)$ :T
    \item $\exists xP(X)$:T
    \item $\forall x\ne 0(P(x))$:F
  \end{enumerate}
\clearpage
\textbf{Problem 2.} [5pts] Let $P(x,y)$ be the statement "$x>y$ and $x$ is divisible by $y$." Let the domain consists of positive integers and write English sentences describing the following propositions.
    \begin{enumerate}[a.]
      \item $\exists x \forall yP(x,y)$
      \item $\exists y \forall xP(x,y)$
      \item $\forall x \exists yP(x,y)$
      \item $\forall y \exists xP(x,y)$
    \end{enumerate}
\textit{\textbf{Solution:}}
    \begin{enumerate}[a.]
      \item $\exists x \forall yP(x,y)$: There exists an element $x$ for all $y$ such that $x>y$ and $x$ is divisible by $y$
      \item $\exists y \forall xP(x,y)$: There exists an element $y$ for all $x$ such that $x>y$ and $x$ is divisible by $y$
      \item $\forall x \exists yP(x,y)$: There exists an element $y$ for all $x$ such that $x>y$ and $x$ is divisible by $y$
      \item $\forall y \exists xP(x,y)$: There exists an element $y$ for all $x$ such that $x>y$ and $x$ is divisible by $y$
    \end{enumerate}
\clearpage
\textbf{Problem 3.} [10pts] Let $P(x,y)$, $Q(x,y)$, and $R(x)$ be propositional functions. Use logical equivalences to show that $\lnot \forall x((\exists y(P(x,y) \implies Q(x,y)) \lor R(x))$ and $\exists x(\lnot R(x) \land \forall y(\lnot Q(x,y) \land P(x,y)))$ are equivalent
\textbf{Solution:}
\begin{center}
  \begin{align*}
    \lnot \forall x((\exists y(P(x,y) \implies Q(x,y)) \lor R(x)) &\equiv \exists x(\lnot(\exists y(P(x,y) \implies Q(x,y)) \lor R(x)))\\
    &\equiv \exists x (\lnot R(x) \land (\lnot \exists y (P(x,y) \implies Q(x,y))))\\
    &\equiv \exists x (\lnot R(x) \land (\forall y (P(x,y) \lor
    Q(x,y))))\\
    &\equiv \exists x (\lnot R(x) \land (\forall y (\lnot Q(x,y) \land P(x,y))))
    the fuck ????
  \end{align*}
\end{center}
\pagestyle{fancyplain}
\chead{DISCRETE STRUCTURE (CO1007) --- SOLUTION to Homework02a --- PREDICATE LOGIC}
\textit{\textbf{\underline{Instruction:}} Type your answers to the following question by LaTeX and submit a zipped file (included.pdf and .tex files) to E-learning site individually. One page per problem. Please use the solution template provided.}
\textit{\textbf{Ref: Rosen's 7th ed - Sec 1.4-1.6}}\\
\textbf{Problem 1.} [5pts] Let P(x) be the statement "$1/x$ =$x$". If the domain consists of real numbers, then what are these truth values?\\
  \begin{enumerate}[a.]
    \item $P(2)$.
    \item $P(1)$.
    \item $\exists xP(X)$
    \item $\forall x\ne 0(P(x))$
  \end{enumerate}
\textit{\textbf{Solution:}}
  \begin{enumerate}[a.]
    \item $P(2)$ :F
    \item $P(1)$ :T
    \item $\exists xP(X)$:T
    \item $\forall x\ne 0(P(x))$:F
  \end{enumerate}
\clearpage
\textbf{Problem 2.} [5pts] Let $P(x,y)$ be the statement "$x>y$ and $x$ is divisible by $y$." Let the domain consists of positive integers and write English sentences describing the following propositions.
    \begin{enumerate}[a.]
      \item $\exists x \forall yP(x,y)$
      \item $\exists y \forall xP(x,y)$
      \item $\forall x \exists yP(x,y)$
      \item $\forall y \exists xP(x,y)$
    \end{enumerate}
\textit{\textbf{Solution:}}
    \begin{enumerate}[a.]
      \item $\exists x \forall yP(x,y)$: There exists an element $x$ for all $y$ such that $x>y$ and $x$ is divisible by $y$
      \item $\exists y \forall xP(x,y)$: There exists an element $y$ for all $x$ such that $x>y$ and $x$ is divisible by $y$
      \item $\forall x \exists yP(x,y)$: There exists an element $y$ for all $x$ such that $x>y$ and $x$ is divisible by $y$
      \item $\forall y \exists xP(x,y)$: There exists an element $y$ for all $x$ such that $x>y$ and $x$ is divisible by $y$
    \end{enumerate}
\clearpage
\textbf{Problem 3.} [10pts] Let $P(x,y)$, $Q(x,y)$, and $R(x)$ be propositional functions. Use logical equivalences to show that $\lnot \forall x((\exists y(P(x,y) \implies Q(x,y)) \lor R(x))$ and $\exists x(\lnot R(x) \land \forall y(\lnot Q(x,y) \land P(x,y)))$ are equivalent
\textbf{Solution:}
\begin{center}
  \begin{align*}
    \lnot \forall x((\exists y(P(x,y) \implies Q(x,y)) \lor R(x)) &\equiv \exists x(\lnot(\exists y(P(x,y) \implies Q(x,y)) \lor R(x)))\\
    &\equiv \exists x (\lnot R(x) \land (\lnot \exists y (P(x,y) \implies Q(x,y))))\\
    &\equiv \exists x (\lnot R(x) \land (\forall y (P(x,y) \lor
    Q(x,y))))\\
    &\equiv \exists x (\lnot R(x) \land (\forall y (\lnot Q(x,y) \land P(x,y))))
    the fuck ????
  \end{align*}
\end{center}
\end{document}

